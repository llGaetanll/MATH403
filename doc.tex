\documentclass[12pt]{article}
\usepackage[
    top=10mm,
    bottom=10mm,
    left=10mm,
    right=10mm,
    marginparwidth=0mm,
    marginparsep=0mm,
    % headheight=15pt,
    centering,
    % showframe,
    includefoot,
    % includehead
]{geometry}
\usepackage{tikz-cd}
\usepackage{multicol}
\usepackage{hyperref}

\hypersetup{
    colorlinks=false,
    linktoc=all,
}

\renewcommand{\date}[1]{\underline{\bf #1}}

\def\eps{\varepsilon}
\def\range{\text{Range}}
\def\proj{\text{proj}}

\newcommand{\GCD}{\text{GCD}}
\def\RowSpace{\text{RowSpace}}
\def\ColSpace{\text{ColSpace}}
\def\trace{\text{trace}}
\def\Ann{\text{Ann}}
\def\sgn{\text{sgn}}
\def\B{\mathcal B}
\def\P{\mathcal P}

\def\Tinv{$T$-invariant}


% create a command for TODOs
\newcommand{\TODO}{\color{red}\textbf{TODO}\color{black}}

\newcommand*{\xdash}[1][3em]{\color{darkgray}\rule[0.5ex]{#1}{0.55pt}\color{black}}

\newcommand{\ip}[2]{\left( #1 \mid #2 \right)}

% preamble
\usepackage{amssymb}
\usepackage{amsmath, amsthm}
\usepackage{xcolor}
\usepackage{fancyhdr}
\usepackage{enumitem}
\usepackage{mathtools}
\usepackage{framed}
\usepackage{parskip}
\usepackage{graphicx}
\usepackage{chngcntr}
\usepackage{float}
\usepackage{listings}
\usepackage{inconsolata}
\usepackage{transparent}
\usepackage{tikz}


% \input xypic (for commutative diagrams)
% \include{mssymb}

\def\A{{\mathbb A}}
\def\P{{\mathbb P}}
\def\N{{\mathbb N}}
\def\Z{{\mathbb Z}}
\def\Q{{\mathbb Q}}
\def\R{{\mathbb R}}
\def\C{{\mathbb C}}
\def\F{{\mathbb F}}
\def\O{{\cal O}}
\let\sec\S
\let\S\relax
\def\S{{\mathfrak S}}
\def\g{{\mathfrak g}}
\def\p{{\mathfrak p}}
\def\h{{\mathfrak h}}
\def\n{{\mathfrak n}}
\def\v{{\mathfrak v}}
\def\m{{\mathfrak m}}
\def\a{{\alpha}}


\newcommand{\skipline}{\vspace{\baselineskip}}
\newcommand{\dis}{\displaystyle}
\newcommand{\noin}{\noindent}


% remove all paragraph indents
\setlength{\parindent}{0pt}

% Figure counter include section
\counterwithin{figure}{section}

% Cleaner figures
\newcommand{\fig}[3][0.4]{
  \begin{figure}[H]
    \centering
    \includegraphics[width=#1\textwidth, keepaspectratio]{#2}
    \caption{#3}
  \end{figure}
}

% Parens, Brackets, Bars, and Braces
\newcommand{\parens}[1]{ \left(#1\right) }
\newcommand{\bracks}[1]{ \left[#1\right] }
\newcommand{\braces}[1]{ \left\{#1\right\} }
\newcommand{\abs}[1]{ \left|#1\right| }
\newcommand{\floor}[1]{ \left\lfloor#1\right\rfloor }
\newcommand{\ceil}[1]{ \left\lceil#1\right\rceil }

% Mathematical notation


\newcommand{\Span}{\mathrm{Span}}
\newcommand{\Range}{\mathrm{Range}}
\newcommand{\Null}{\mathrm{Null}}
\newcommand{\Rank}{\mathrm{Rank}}
\newcommand{\rank}{\mathrm{rank}}
\newcommand{\Nullity}{\mathrm{Nullity}}
\newcommand{\nullity}{\mathrm{nullity}}
\newcommand{\longhookrightarrow}{\lhook\joinrel\relbar\joinrel\rightarrow}
\newcommand{\la}{\leftarrow}
\newcommand{\ra}{\rightarrow}
\newcommand{\La}{\Leftarrow}
\newcommand{\Ra}{\Rightarrow}
\newcommand{\dbar}{\overline{\partial}}
\newcommand{\gequ}{\geqslant}
\newcommand{\lequ}{\leqslant}
\newcommand{\Hom}{\mathrm{Hom}}
\newcommand{\End}{\mathrm{End}}
\newcommand{\Aut}{\mathrm{Aut}}
\newcommand{\Coker}{\mathrm{Coker}}
\newcommand{\Row}{\mathrm{Row}}
\newcommand{\Ker}{\mathrm{Ker}}
\newcommand{\Tr}{\mathrm{Tr}}
\newcommand{\Id}{\mathrm{Id}}
% \newcommand{\mod}{\mathrm{mod }}
\newcommand{\un}{\underline}
\newcommand{\ov}{\overline}
\newcommand{\wt}{\widetilde}
\newcommand{\wh}{\widehat}
\newcommand{\pr}{\prime}
\newcommand{\rk}{\mathrm{rk}}
\newcommand{\im}{\mathrm{Im}}

% Linear Algebra

\newcommand{\lind}{linearly independent}
\newcommand{\ldep}{linearly dependent}
\renewcommand{\vec}[1]{
  {\bf #1}
}
\newcommand{\lincomb}[3]{
  #1_{1}#2_{1} + #1_{2}#2_{2} + \cdots + #1_{#3}#2_{#3}
}
\newcommand{\neglincomb}[3]{
  -#1_{1}#2_{1} - #1_{2}#2_{2} - \cdots - #1_{#3}#2_{#3}
}
\newcommand{\constants}[2]{
  #1_{1}, #1_{2}, \cdots, #1_{#2}
}
\newcommand{\constantsz}[2]{
  #1_{0}, \constants{#1}{#2}
}

% Analysis
\newcommand{\limfty}[1]{\lim_{#1 \to \infty}}
\newcommand{\seq}[2]{\{#1_{#2}\}_{#2 \in \N}}
\newcommand{\sseq}[3]{\{#1_{#2_{#3}}\}_{#3 \in \N}}
\newcommand{\chep}{Let $\epsilon > 0$}

% Category Theory
\newcommand{\catC}{\mathcal{C}}
\newcommand{\catD}{\mathcal{D}}
\newcommand{\functF}{\mathcal{F}}
\newcommand{\functG}{\mathcal{G}}
\newcommand{\functI}{\mathcal{I}}
\newcommand{\functU}{\mathcal{U}}

\newcommand{\op}[1]{#1^{\mathrm{op}}}
\newcommand{\Obj}{\mathrm{Obj}}

\newcommand{\Set}{\mathbf{Set}}
\newcommand{\Grp}{\mathbf{Grp}}
\newcommand{\Top}{\mathbf{Top}}
\newcommand{\cRing}{\mathbf{cRing}}
\newcommand{\BanAnaMan}{\mathbf{BanAnaMan}}
\newcommand{\FinSet}{\mathbf{FinSet}}
\newcommand{\Vect}{\mathbf{Vect}}
\newcommand{\Two}{\mathbf{2}}


% ================= %
% Headers & Footers
% ================= %
\pagestyle{fancy}
\fancyhf{}
\newcommand{\intros}[3]{
  \lhead{\textbf{#1} {#2}}
  \rhead{#3}}
\rfoot{\thepage}
\renewcommand{\headrulewidth}{0pt}



% ================= %
%       Utils
% ================= %
\newcommand{\induction}[3]{
  \textbf{Base Case}. #1 \\
  \textbf{Inductive Hypothesis}. \\ #2 \\
  \textbf{Inductive Step}. \\ #3
}



% Used to list all problems on homework
\newcommand{\problems}[1]{
  \medskip \noin
  {\bf Problems}

  #1

  \medskip{}
}


% When prof does Question/Answer styling
\newcommand{\qna}[2]{
  {\bf Question}: #1

  {\bf Answer}: #2
}


% augmented matrices
\makeatletter
\renewcommand*\env@matrix[1][*\c@MaxMatrixCols c]{%
  \hskip -\arraycolsep
  \let\@ifnextchar\new@ifnextchar
  \array{#1}}
\makeatother


\newcommand\restr[2]{{% we make the whole thing an ordinary symbol
  \left.\kern-\nulldelimiterspace % automatically resize the bar with \right
  #1 % the function
  \littletaller % pretend it's a little taller at normal size
  \right|_{#2} % this is the delimiter
  }}

\newcommand{\littletaller}{\mathchoice{\vphantom{\big|}}{}{}{}}

% ================= %
%      Box Meta
% ================= %

% #2 - FG Color
% #3 - BG Color
\newenvironment{fancyleftbar}[3][\hsize]
{%
    \def\FrameCommand
    {%
        {\color{#2}\vrule width 3pt}%
        \hspace{0pt}%must no space.
        \fboxsep=\FrameSep\colorbox{#3}%
    }%
    \MakeFramed{\hsize#1\advance\hsize-\width\FrameRestore}%
}
{\endMakeFramed}

\newenvironment{simpleleftbar}[3][\hsize]
{%
    \def\FrameCommand
    {%
        {\vrule width 0.5pt}%
        \hspace{3pt}
        \fboxsep=\FrameSep%
    }%
    \MakeFramed{\hsize#1\advance\hsize-\width\FrameRestore}%
}
{\endMakeFramed}

% Used to allow the color argument to pass through the environment%
\newsavebox{\boxqed} 

% #1 - Header
% #2 - FG Color
% #3 - BG Color
\newenvironment{fancybox}[3]{
  \sbox\boxqed{\textcolor{#2}{$\blacksquare$}}
  \begin{fancyleftbar}{#2}{#3}

  \noin
  #1
  % {\large \bf \underline{#1}}
  \smallskip\noin \\
}
{

  \medskip
  \noin
  \usebox\boxqed

  \end{fancyleftbar}
}

% #1 - Text header
% #2 - Outer Text
% #3 - Inner Text
% #4 - Inner Header
% #5 - FG Color
% #6 - Background Color
\newcommand{\boxmeta}[6]{
  #1
  % {\small\sc\uppercase{#1}}

  #2

  \begin{fancybox}{#4}{#5}{#6}
    \noin
    #3
  \end{fancybox}
}

% #1 - Title
% #2 - FG Color
% #3 - BG Color
% #4 - Inner Text
\newcommand{\baronly}[4]{
  \begin{simpleleftbar}{#2}{#3}
    {\bf #1}.

    #4
  \end{simpleleftbar}
}

% ================= %
%     Box Colors
% ================= %

\definecolor{theorem_fg}{HTML}{EABAC3}
\definecolor{theorem_bg}{HTML}{F9EEF0}

\definecolor{problem_fg}{HTML}{ABABAB}
\definecolor{problem_bg}{HTML}{EDEDED}

\definecolor{lemma_fg}{HTML}{D0C97D}
\definecolor{lemma_bg}{HTML}{FCF9DB}

\definecolor{prop_fg}{HTML}{7DDB89}
\definecolor{prop_bg}{HTML}{D7FADB}

\definecolor{defn_fg}{HTML}{83D4CF}
\definecolor{defn_bg}{HTML}{E7FCFB}

\definecolor{lst_bg}{HTML}{EFF6F8}
\definecolor{lst_fg}{HTML}{475857}

\definecolor{btw_fg}{HTML}{5A5A5A}

% ================= %
%     Box Envs
% ================= %

\newcommand{\Definition}[2]{
                              \boxmeta{}{}{#2}{{\it Definition}. {\bf\underline{#1}}}{defn_fg}{defn_bg}
                            }

\newcommand{\Theorem}[2]{
  \boxmeta{{\bf Theorem.}}{#1}{#2}{{\bf Proof.}}{theorem_fg}{theorem_bg}
}

\newcommand{\NamedTheorem}[3]{
  \boxmeta{#1}{#2}{#3}{Proof}{theorem_fg}{theorem_bg}
}

\newcommand{\Problem}[3]{
  \boxmeta{Problem #1}{#2}{#3}{Solution}{problem_fg}{problem_bg}
}

\newcommand{\Example}[2]{
  \boxmeta{Example}{#1}{#2}{}{problem_fg}{problem_bg}
}

\newcommand{\Lemma}[2]{
  \boxmeta{{\bf Lemma.}}{#1}{#2}{{\bf Proof.}}{lemma_fg}{lemma_bg}
}

\newcommand{\NamedLemma}[3]{
  \boxmeta{#1}{#2}{#3}{Proof}{lemma_fg}{lemma_bg}
}

\newcommand{\Corollary}[2]{
  \boxmeta{Corollary}{#1}{#2}{Proof}{lemma_fg}{lemma_bg}
}

\newcommand{\Proposition}[2]{
  \boxmeta{Proposition}{#1}{#2}{Proof}{prop_fg}{prop_bg}
}

% ================== %
%      Bar Only
% ================== %

\newcommand{\definition}[1]{
  \baronly{Definition}{defn_fg}{defn_bg}{#1}
}

\newcommand{\theorem}[1]{
  \baronly{Theorem}{theorem_fg}{theorem_bg}{#1}
}

\newcommand{\example}[1]{
  \baronly{Example}{problem_fg}{problem_bg}{#1}
}

\newcommand{\remark}[1]{
  \baronly{Remark}{problem_fg}{problem_bg}{#1}
}

\newcommand{\note}[1]{
  \medskip
  \baronly{Note}{problem_fg}{problem_bg}{#1}
}

\newcommand{\btw}[1]{
  {\bf Author Note}.

  \begin{color}{btw_fg}
    #1
  \end{color}
}

\newcommand{\sidenote}[1]{
  {\bf Side Note}.

  \boxmeta{}{}{#1}{}{problem_fg}{problem_bg}
}



% graphs
\def\deg{\text{deg}}
\def\indeg{\text{indeg}}
\def\outdeg{\text{outdeg}}

% big O notation
\def\O{\mathcal O}

% define parent
\def\pr{\text{pr}}

% define incomplete commands
\def\TODO{\color{red}\textbf{TODO}\color{black}\,}
\def\QUESTION{\color{red}\textbf{QUESTION}\color{black}\,}



% listings settings
\lstset{
  % general styles
  backgroundcolor=\color{lst_bg},
  numbers=left,
  numberstyle=\color{lst_fg}\ttfamily\textbf,
  numbersep=3mm,
  frame=l,
  framesep=7mm,
  framexleftmargin=1.5mm,
  fillcolor=\color{lst_bg},
  rulecolor=\color{lst_bg},
  xleftmargin=9mm,
  % keyword styles
  keywordstyle=[1]\textbf,
  keywordstyle=[2]\textit,
  keywordstyle=[3]\textbf\textit,
  keywords=[1]{let, for, while, not, if, else, then, do, end, return},
  keywords=[2]{if, condition},
  keywords=[3]{do},
  mathescape=true, % enable math mode in listings
  columns=fullflexible,
  basicstyle=\ttfamily % this fond looks a bit better than the default
}





% colors used in proof boxes
\definecolor{proof_fg}{HTML}{ABABAB}
\definecolor{proof_bg}{HTML}{EDEDED}



% box solutions
\newenvironment{solution}{
  \begin{leftbar}

  \noin
  {\large \sc solution} \\
}
{

  \medskip
  \noin
  \textcolor{proof_fg}{$\blacksquare$}

  \end{leftbar}
}



% style a problem and solution
\newcommand{\prob}[3]{
  \bigskip \bigskip \noin
  {\Large\sc Problem #1}

  \medskip\noin
  #2

  \begin{solution}
    \noin
    #3
  \end{solution}
}

\renewcommand{\mod}[1]{
  \text{ (mod $#1$) }
}

\def\mat{\text{Mat}}
\def\gl{\text{GL}}
\def\sl{\text{SL}}



\begin{document}

\tableofcontents

\date{Wed. 24 Jan 2024}

\note {
All info for the class is available on the canvas page.
Notes from the prof are written on the iPad, and PDFs will be provided after
each class. Despite this taking notes is helpful.

Offices are hours are Tuesdays in person, and Thursdays on Zoom. Hours may
vary.

A text for this class is not {\it required}. Technically, we are using
Fraleigh's {\it A first course in Abstract Algebra}

No quizzes in this class, weekly Homeworks except on Exam weeks, two
midterms, and one final.

Readings on the class schedule are not additional, it's for people that need
extra material, or people that missed that day.
}

\section{Introduction - Sets and Relations}

\subsection{Sets}

\definition{
	A {\bf Set}. is a well-defined collection of objects called {\it elements}.

	$a \in A$ means ``$A$ is a set, $a$ is an element of a set, and $a$ is in
	$A$.
}

Examples of sets are

\begin{itemize}
	\item $\Z$ - The set of all integers, positive, negative, and zero
	\item $\N$ - The set of natural numbers, $0, 1, 2, \dots$. {\bf In this
			      class, $\N$ starts with $1$.}
	\item $\Q$ - The set of rational numbers.
	\item $\R$ - The set of real numbers.
	\item $\C$ - The set of complex numbers.
	\item $\{1, 2, 3, 4\}$
	\item $\{a \in \Z | a > 2 \}$. This is a set of integers {\it such that} $a
		      > 2$.
	\item $\varnothing$ - The empty set.
	\item $\text{GLn}(\R)$ - The set of $n \times n$ invertible matrices with real
	      entries. (GL stands for ``General Linear".)
	\item $C(\R)$ - The set of continuous functions $f: \R \to \R$.
\end{itemize}

\definition{
	A set $A$ is a {\bf subset} of a set $B$ if

	\[
		\forall x \in A, x \in B
	\]

	In other words, everything in $A$ is also in $B$. As notation, we can say
	either $A \subseteq B$ or $A \subset B$.

	A {\bf proper} subset is $A \subset B$ but $A \ne B$. Just write $A
		\subsetneq B$.
}

For example, $\N \subseteq \Z \subseteq \Q \subseteq \R \subseteq \C$. And,
importantly, $\varnothing \subseteq A$ for all sets $A$. In other words, the
empty set is a subset of {\it all} sets.

Two sets are equal if $A = B$, or $A \subseteq B$ {\it and} $B \subseteq A$.
This is often how you prove set equality.

\subsubsection{Set Operations}

We have four main operations.

\begin{itemize}
	\item {\bf Union}: $A \cup B = \{x | x \in A \text{ or } x \in B \}$.
	\item {\bf Intersection}: $A \cap B = \{ x | x \in A \text{ and } x \in B \}$.
	\item {\bf Product}: $A \times B = \{(a, b) | a \in A, b \in B \}$.
	\item {\bf Difference}: $A \setminus B = \{x | x \in A \text{ and } x \not\in B\}$
\end{itemize}

Two sets are {\bf disjoint} if $A \cap B = \varnothing$.

If we're working in a particular {\it universe} $U$ (i.e. all sets are subsets
of the universal set $U$) then the {\it complement} of $A$ is $A^c = \{x | x
	\in U \text{ and } x \not\in A \}$.

\subsection{Relations}

\definition {
	A {\bf relation} between sets $A$ and $B$ is a subset $R \subseteq A \times
		B$.
}

If $(a, b) \in R$, then we say that ``$a$ is related to $b$", or we write
$aRb$, or $a \sim b$.

\example {
	$R \subseteq \{1, 2, 3\} \times \{2, 3, 4\}$. $R = \{(1, 3), (2, 2), (3, 4) \}$
}

\note {
	Relations might not be reflexive! If $(a, b) \in R$, that means $a$ is
	related to $b$, but it might not be the case that $(b, a) \in R$. In other
	words, the reverse may not be true!
}

Another example might be $R \subseteq \R \times \R$, with $R = \{(x, x^3) | x
	\in \R \}$. Oh look! We just rewrote $f(x) = x^3$, so functions are relations.

\definition {
	A {\bf partition} of a set $A$ is a collection of {\it disjoint} subsets
	whose union is $A$.

	Another way to think of this is that any element of $A$ is in one and only
	one of its partitions.
}

An example of this might be the partition

\[
	A = \Z = \{x \in \Z | x < 0 \} \cup \{ 0 \} \cup \{ x \in \Z | x > 0 \}
\]

is a partition of $\Z$ into 3 sets.

Another example might be $A = \R$, subsets are $\{ x \}$ for each $x \in \R$.

Another, maybe more interesting example might be the following.

\example {
Fix $n \in \N$, $n \ge 2$. Let

\begin{itemize}
	\item $\bar 0 = \{x \in \Z | x \text{ is divisible by } n \}$.
	\item $\bar 1 = \{x \in \Z | x - 1 \text{ is divisible by } n \}$.
	\item $\bar 2 = \{x \in \Z | x - 2 \text{ is divisible by } n \}$. On and
	      on until...
	\item $\overline{n - 1} = \{x \in \Z | x - (n - 1) \text{ is divisible by } n
		      \}$.
\end{itemize}

{\bf Claim}: This partitions $\Z$ into in $n$ subsets.

}

\date{Fri. 26 Jan 2024}

Let's go back to relations, which we put aside to talk about partitions.

\definition {
	A relation $A \subseteq A \times A$ is called an {\bf equivalence relation}
	if it satisfies 3 properties

	\begin{enumerate}
		\item {\bf Reflexivity}: $aRA$ for all $a \in A$.
		\item {\bf Symmetry}: $aRb$ if and only if $bRa$.
		\item {\bf Transitivity}: If $aRb$ and $bRc$, then $aRc$.
	\end{enumerate}
}

The key idea is that equivalence relations on $A$ are {\it the same} as
partitions of $A$. What's going on here?

From an equivalence relation: If $b$ is related to $b$, put them in the same
set. Because of symmetry of equivalence relations, order of elements in the
set doesn't matter.

Conversely, given a partition say $aRb \Leftrightarrow bRa$ are in the same
subset.

\note {
	We'll be talking a lot about partitions and equivalence relations in this
	class.
}

Now we move on to the next step in our intro: functions.

\subsection{Functions}

A function $f: A \to B$ is a relation $R_f \subseteq A \times B$ such that,
for all $a \in A$, there is a {\it unique} $b \in B$ such that $aRb$.
Effectively, this means that

\begin{enumerate}
	\item We pass the vertical line test.
	\item The function is defined over its entire domain.
\end{enumerate}

Which are the properties we expect of functions!

\example {
	Let $f: \R \to \R$, with $R_f = \{(x, x^3) | x \in \R \}$. We write $f(a)$
	for the value $b$ where $(a, b) \in R_f$.
}

Given a function $f: A \to B$. We say that

\begin{itemize}
	\item $A$ is the {\it domain}.
	\item $B$ is the {\it codomain}.
	\item The {\it range} is a {\it subset} of the codomain, only where $f$
	      outputs values.
\end{itemize}

The $+$ operation is a function $+: \R \to \R$, also written as $(a, b)
	\mapsto a + b$. The multiplication operation $\times: \R \times \R \to \R$,
also written as $(a, b) \to ab$. These are binary operations, very useful in
Group Theory.

\definition {
	A {\bf binary operation} on a set $A$ is a function $f: A \times A \to A$.
	Its an operation on two inputs that outputs one thing.
}

\note {
	A dot product does not count here! Because the output of the dot
	product does not come from the same set as the input.
}

To do more complicated things in real life (such as $a + b + c$), we must
parenthesize.

\[
	f(a, f(b, c)) \text{ or } f(f(a, b), c)
\]

Of course this doesn't matter for addition in particular, but it might for
other binary operators!

\example {
	Fix $n \ge 2$ and consider $\Z_n = \{\bar 0, \bar 1, \dots, \overline{n - 1}
		\}$ (Note that this is a set of sets!)

	We want to come up with binary operations on this set. We have

	\begin{enumerate}
		\item Addition: $+: \Z_n \times \Z_n \to \Z_n$, defined as $(\bar a, \bar
			      b) \mapsto \overline{a + b}$.

		      But this isn't well-defined! For instance, what happens if $a + b$
		      exceeds $n$? To fix this, let's add the following condition:

		      Let $\bar x = \bar y$ if $x, y$ are in the same subset of partitions.
		      (i.e. They have the same remainder mod $n$.)

		      \qna {
			      Is this a well-defined binary operations?
		      }
		      {
			      Yes! But we must check that it doesn't matter how we define our
			      inputs.
		      }

		\item Multiplication: $\times: \Z_n \times \Z_n \to \Z_n$, defined as
		      $(\bar a, \bar b) \mapsto \overline{ab}$
	\end{enumerate}
}

\note {
	We also write $x \equiv y \mod{n} $ if $\bar x = \bar y$.
}

Now we're ready to jump in.

\section{Properties of Operations on R}

We know that $+$ has the following properties

Properties of $+$ on $\R$, and $\times$ on $\R \setminus \{0\}$.

\begin{enumerate}
	\item {\bf Associativity}: $a + (b + c) = (a + b) + c$, and similarly for
	      multiplication

	\item {\bf Identity}: $a + 0 = a$, $a \cdot 1 = a$.

	\item {\bf Inverses}: $a + (-a) = (-a) + a = 0$

	\item {\bf Commutativity}: $a + b = b + a$
\end{enumerate}

\TODO{} do this for multiplication.

\definition {
	We say that a binary operation $p: A \times A \to A$ is {\bf associative} if

	\[
		p(a, p(b, c)) = p(p(a, b), c)
	\]

	for any $a, b, c \in A$. In other words, how we parenthesize doesn't matter.
}

\definition {
	We say that a binary operation $p: A \times A \to A$ has an {\bf identity} if

	\[
		p(a, e) = p(e, a) = a
	\]

	for any $a \in A$.
}

\definition {
	We say that a binary operation $p: A \times A \to A$ has a {\bf inverses} if

	\begin{enumerate}
		\item It has an identity element $e$ (otherwise identity is meaningless!)

		\item
		      \[
			      p(a, b) = p(b, a) = e
		      \]

		      for any $a \in A$ and some $b \in A$.
	\end{enumerate}

	We usually write $b$ as $a^{-1}$.
}

\definition {
	We say that a binary operation $p: A \times A \to A$ is {\bf commutative} if

	\[
		p(a, b) = p(b, a)
	\]

	for any $a, b \in A$.
}

Let's look at properties of $(\Z_n, +)$ and $(\Z_n, \times)$.

$(\Z_n, +)$

\begin{enumerate}
	\item Is Associative.
	\item Has an identity: $0$.
	\item Has an inverse: $\overline{(-a)}$ for any $\bar a$.
	\item Is Commutative: We can move elements around.
\end{enumerate}

$(\Z_n, \times)$

\begin{enumerate}
	\item Is Associative
	\item Has an identity: $1$.
	\item {\bf Does not} have an inverse! Because $\bar 0$ is still there, we
	      have no inverse.
	\item Is Commutative: We can move elements around. In this case, $\bar a
		      \bar b = \overline{ab} = \overline{ba} = \bar b \bar a$.
\end{enumerate}

\qna {
	If we instead looked at $(\Z_n \setminus \{ 0 \}, \times)$, would there be
	inverses?
}
{
	We've messed the whole thing up! This is not even an binary operation
	anymore. Since we don't have $\bar 0$, what does $\bar 2 + \bar 2$ even mean
	now, if $n = 4$?

	We'll study this more in about a week.
}

Let's look at matrices. $A = \mat_n(\R)$ be the set of $n \times n$ matrices
with real elements, with the binary operation being matrix multiplication.
Let's look at its properties.

\begin{enumerate}
	\item Its associative.
	\item It has an identity.
	\item It {\bf does not} have an inverse.
	\item It {\bf is not} commutative.
\end{enumerate}

Now looking at $A = \gl_n(\R)$ be the set of $n \times n$ invertible matrices
with real entries.

\begin{enumerate}
	\item Its associative.
	\item It has an identity.
	\item It {\bf does} have an inverses.
	\item It {\bf is not} commutative.
\end{enumerate}

{\bf Proposition}

If $p: A \times A \to A$ is a binary operation with two identities $e, f$,
then $e = f$.

	{\bf Proof}

$e = p(e, f) = f$, so $e = f$.


	{\bf Proposition}

If we have two inverses, then they are the same. More formally: if $p(a, b) =
	p(b, a)= e$, and $p(a, c) = p(c, a) = e$, then $b = c$

{\bf Proof}

$p(c, p(a, b)) = p(c, e) = c$, but we could have also done $p(p(c, a), b) =
	p(e, b) = b$, so $b = c$.

\date{Mon. 29 Jan 2024}

\note {
	Homework 1 is due this Thursday at 11:59PM, on Gradescope. Because this is
	the first homework, Gradescope will allow late submissions but just submit
	it on time.
}

Last time, we talked about binary operations and their properties. Now, we are
going to put everything together and talk about Groups!

\section{Groups}

\definition {
	A {\bf Group} is a set $G$ with a binary operation $p: G \times G \to G$
	that

	\begin{enumerate}
		\item Is {\it Associative}.
		\item Has an {\it Identity}.
		\item Has {\it Inverses}.
	\end{enumerate}
}

Note that it does {\bf not} have commutatitivity. We'll talk about that
later.

Notation-wise, we write $(G, p)$, or just $G$ if the binary operations is
understood. Additionally, we often write the operation as $a \cdot b$, $a + b$,
or $ab$ instead of $p(a, b)$.

\definition {
  A Group is {\bf Abelian} if the operation is also {\it commutative}.
}

\note {
  Sometimes, we say that a group is {\it closed} under its operation. However
  we don't need this because a binary operation, by definition, is necessarily
  closed.
}

Let's look at some examples.

\example {
  These groups are {\bf Abelian}:
  \begin{enumerate}
    \item $(\R, +)$
    \item $(\Z, +)$
    \item $(\C, +)$
    \item $(\R \setminus \{0\}, \times)$
  \end{enumerate}

  These groups are {\bf Non-Abelian}:
  \begin{enumerate}
    \item $(\gl_n(\R), \times)$. Recall that this is the set of {\it
      non-invertible} $n \times n$ matrices with real entries.
    \item $(\Z_n, +)$. Recall that this was the set of classes of partitions
      modulo $n$.
  \end{enumerate}

  These are {\bf not Groups}:
  \begin{enumerate}
    \item $(\N \cup \{0\}, +)$, has no inverses.
    \item $\mat_n(\R), \times)$, has no inverses.
  \end{enumerate}
}

\definition {
  The {\bf Order} of a group, is the {\it cardinality} of the set $G$, denoted
  $|G|$.
}

\subsection{Cancellation Law}

In a group $G$, if $ab = ac$, then $b = c$.

{\bf Proof}

Since $G$ has inverses, there is an element $a^{-1} \in G$ such that $a a^{-1} =
e$. So,

\begin{align*}
  ab &= ac \\
  a^{-1} (ab) &= a^{-1} (ac) \\
  (a^{-1}a) b &= (a^{-1}a) c \\
  b &= c & \text{Defn of Identity}\\
\end{align*}

Let's look at some more examples.

\subsection{Groups of Matrices}

\example {
  From the groups of matrices, we can also talk about

  \begin{itemize}
    \item $\gl_n(\R)$
    \item $\gl_n(\C)$
    \item $\gl_n(\Q)$
  \end{itemize}

  Which are all groups.

  {\bf Question}: Is $\gl_n(\N)$ a group? What about $\gl_n(\Z)$?
}

Recall that $\gl$ stands for {\it general linear}. There is also the {\it
special linear} group $\sl$. This is the set of general linear matrices with
determinant $1$. Let's look at some examples

\example {
  \begin{enumerate}
    \item $\sl = \{ A \in \gl_n(\R) | \det(A) = 1\}$

      Recall that $\det(AB) = \det(A) \det(B)$, so this is closed.
  \end{enumerate}
}

\subsection{Symmetric Groups}

\definition {
  Given the set $\{1, 2, \dots, n\}$, the group of {\it permutations} of this
  set is the {\bf symmetric group} $S_n$, where the binary operation is {\it
  function composition}.

  A {\bf permutation} is a bijection $\{1, 2, \dots, n\ \to \{1, 2, \dots, n\}$.
  A permutation can be described by a list.
}

\example {
  If $n = 3$, we have the permutations

  \[
    \{123, 213, 132, 321, 231, 312 \}
  \]

  We say that $\sigma: \{1, 2, \dots, n\} \to \{1, 2, \dots, n\}$ takes an input
  from the set and defines the shuffle.
}

We'll talk more about Symmetric groups later in the semester.

\note {
  \begin{itemize}
    \item The Symmetric group is {\bf Non-Abelian}.
    \item There are $n!$ permutations of $S_n$, so the order of $S_n$ is $|S_n|
      = n!$
    \item Why the Symmetric Group is named as it is is a question for another
      day.
  \end{itemize}
}

\subsection{Subgroups}

\definition {
  A {\bf subgroup} is a subset $H$ of a group $(G, p)$ such that:

  \begin{enumerate}
    \item $H$ is closed under $p$.

      If $a, b \in H$, then $p(a, b) \in H$.

      Note that we {\it need} to explicitly state that a subgroup is closed,
      because $p$ is {\bf not} closed in $H$, but it {\it is} by virtue of the
      values in $H$. However this does not come for free from $p$, unlike with
      $G$ like before.

    \item $H$ has inverses.

      If $a \in H$, then $a^{-1} \in H$.
  \end{enumerate}

  Note that we also have

  \begin{enumerate}
    \item {\bf The Identity}, by virtue of $H$ being closed and containing
      inverses.

      \[
        a a^{-1} = e
      \]

    \item {\bf Associativity}, because $(G, p)$ is associative. This property is
      just inherited from $G$.
  \end{enumerate}

  So $(H, p)$ is a group!
}

As notation, we say that $H \le G$ if $H$ is a subgroup of $G$.

\example {
  \[
    \sl_n(\R) \le \gl_n(\R) \le \gl_n(\C)
  \]

  Notice the direction of ``subset-ness"!
}

\example {
  \[
    \{\bar 0, \bar 2\} \le (\Z_4, +)
  \]

  Let's check this one.

  \begin{enumerate}
    \item {\bf Closure}:

      This is small enough that we can check them all.

      \begin{itemize}
        \item $\bar 0 + \bar 0 = \bar 0$
        \item $\bar 0 + \bar 2 = \bar 2$
        \item $\bar 2 + \bar 0 = \bar 2$
        \item $\bar 2 + \bar 2 = \bar 4 = \bar 0$
      \end{itemize}

      Note that we didn't really need to check the middle two, since the group
      is Abelian, and that property is inherited.

    \item {\bf Inverses}

      \begin{itemize}
        \item $\bar 0 + \bar 0 = \bar 0$
        \item $\bar 2 + \bar 2 = \bar 0$
      \end{itemize}
  \end{enumerate}

  So we have a subgroup!
}

\example {
  If $G$ is a group and $a \in G$ is an element, then

  \[
    H = \{ \dots, a^{-3}, a^{-2}, a^{-1}, e, a^1, a^2, a^3, \dots \}
  \]

  is a subgroup.

  As a note

  \begin{itemize}
    \item $a^{-3} = a^{-1}a^{-1}a^{-1}$
    \item $a^2 = aa$
  \end{itemize}
}

Furthermore, sometimes, $a^n = e$ for some finite $n$. The smallest such $n$ is
called the {\bf order} of $a$.

\example {
  The order of $\bar 2 \in \Z_4$ is $2$.
}

\definition {
  We say that a subgroup $H \le G$ is called {\bf trivial} if $|H| = 1$. Or,

  \[
    H = \{e\}
  \]

  This is a subgroup of {\it every group}.
}

\note {
  $G \le G$ for all groups $G$. In other words, a group is always a subgroup of
  itself.

  We can say that $H < G$ is a {\bf proper} subgroup if $H \le G$ but $H \ne G$
  and, additionally for this class, $H$ is non-trivial.
}

\end{document}
