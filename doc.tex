\documentclass[12pt]{article}
\usepackage[
    top=10mm,
    bottom=10mm,
    left=10mm,
    right=10mm,
    marginparwidth=0mm,
    marginparsep=0mm,
    % headheight=15pt,
    centering,
    % showframe,
    includefoot,
    % includehead
]{geometry}
\usepackage{tikz-cd}
\usepackage{multicol}
\usepackage{hyperref}

\hypersetup{
    colorlinks=false,
    linktoc=all,
}

\renewcommand{\date}[1]{\underline{\bf #1}}

\def\eps{\varepsilon}
\def\range{\text{Range}}
\def\proj{\text{proj}}

\newcommand{\GCD}{\text{GCD}}
\def\RowSpace{\text{RowSpace}}
\def\ColSpace{\text{ColSpace}}
\def\trace{\text{trace}}
\def\Ann{\text{Ann}}
\def\sgn{\text{sgn}}
\def\B{\mathcal B}
\def\P{\mathcal P}
\newcommand{\lcm}{\text{lcm}}

\def\phi{\varphi}
\def\sgn{\text{sgn}}

\def\Tinv{$T$-invariant}


% create a command for TODOs
\newcommand{\TODO}{\color{red}\textbf{TODO}\color{black}}

\newcommand*{\xdash}[1][3em]{\color{darkgray}\rule[0.5ex]{#1}{0.55pt}\color{black}}

\newcommand{\ip}[2]{\left( #1 \mid #2 \right)}

\newcommand{\lr}[1]{\langle #1 \rangle}

% preamble
\usepackage{amssymb}
\usepackage{amsmath, amsthm}
\usepackage{xcolor}
\usepackage{fancyhdr}
\usepackage{enumitem}
\usepackage{mathtools}
\usepackage{framed}
\usepackage{parskip}
\usepackage{graphicx}
\usepackage{chngcntr}
\usepackage{float}
\usepackage{listings}
\usepackage{inconsolata}
\usepackage{transparent}
\usepackage{tikz}


% \input xypic (for commutative diagrams)
% \include{mssymb}

\def\A{{\mathbb A}}
\def\P{{\mathbb P}}
\def\N{{\mathbb N}}
\def\Z{{\mathbb Z}}
\def\Q{{\mathbb Q}}
\def\R{{\mathbb R}}
\def\C{{\mathbb C}}
\def\F{{\mathbb F}}
\def\O{{\cal O}}
\let\sec\S
\let\S\relax
\def\S{{\mathfrak S}}
\def\g{{\mathfrak g}}
\def\p{{\mathfrak p}}
\def\h{{\mathfrak h}}
\def\n{{\mathfrak n}}
\def\v{{\mathfrak v}}
\def\m{{\mathfrak m}}
\def\a{{\alpha}}


\newcommand{\skipline}{\vspace{\baselineskip}}
\newcommand{\dis}{\displaystyle}
\newcommand{\noin}{\noindent}


% remove all paragraph indents
\setlength{\parindent}{0pt}

% Figure counter include section
\counterwithin{figure}{section}

% Cleaner figures
\newcommand{\fig}[3][0.4]{
  \begin{figure}[H]
    \centering
    \includegraphics[width=#1\textwidth, keepaspectratio]{#2}
    \caption{#3}
  \end{figure}
}

% Parens, Brackets, Bars, and Braces
\newcommand{\parens}[1]{ \left(#1\right) }
\newcommand{\bracks}[1]{ \left[#1\right] }
\newcommand{\braces}[1]{ \left\{#1\right\} }
\newcommand{\abs}[1]{ \left|#1\right| }
\newcommand{\floor}[1]{ \left\lfloor#1\right\rfloor }
\newcommand{\ceil}[1]{ \left\lceil#1\right\rceil }

% Mathematical notation


\newcommand{\Span}{\mathrm{Span}}
\newcommand{\Range}{\mathrm{Range}}
\newcommand{\Null}{\mathrm{Null}}
\newcommand{\Rank}{\mathrm{Rank}}
\newcommand{\rank}{\mathrm{rank}}
\newcommand{\Nullity}{\mathrm{Nullity}}
\newcommand{\nullity}{\mathrm{nullity}}
\newcommand{\longhookrightarrow}{\lhook\joinrel\relbar\joinrel\rightarrow}
\newcommand{\la}{\leftarrow}
\newcommand{\ra}{\rightarrow}
\newcommand{\La}{\Leftarrow}
\newcommand{\Ra}{\Rightarrow}
\newcommand{\dbar}{\overline{\partial}}
\newcommand{\gequ}{\geqslant}
\newcommand{\lequ}{\leqslant}
\newcommand{\Hom}{\mathrm{Hom}}
\newcommand{\End}{\mathrm{End}}
\newcommand{\Aut}{\mathrm{Aut}}
\newcommand{\Coker}{\mathrm{Coker}}
\newcommand{\Row}{\mathrm{Row}}
\newcommand{\Ker}{\mathrm{Ker}}
\newcommand{\Tr}{\mathrm{Tr}}
\newcommand{\Id}{\mathrm{Id}}
% \newcommand{\mod}{\mathrm{mod }}
\newcommand{\un}{\underline}
\newcommand{\ov}{\overline}
\newcommand{\wt}{\widetilde}
\newcommand{\wh}{\widehat}
\newcommand{\pr}{\prime}
\newcommand{\rk}{\mathrm{rk}}
\newcommand{\im}{\mathrm{Im}}

% Linear Algebra

\newcommand{\lind}{linearly independent}
\newcommand{\ldep}{linearly dependent}
\renewcommand{\vec}[1]{
  {\bf #1}
}
\newcommand{\lincomb}[3]{
  #1_{1}#2_{1} + #1_{2}#2_{2} + \cdots + #1_{#3}#2_{#3}
}
\newcommand{\neglincomb}[3]{
  -#1_{1}#2_{1} - #1_{2}#2_{2} - \cdots - #1_{#3}#2_{#3}
}
\newcommand{\constants}[2]{
  #1_{1}, #1_{2}, \cdots, #1_{#2}
}
\newcommand{\constantsz}[2]{
  #1_{0}, \constants{#1}{#2}
}

% Analysis
\newcommand{\limfty}[1]{\lim_{#1 \to \infty}}
\newcommand{\seq}[2]{\{#1_{#2}\}_{#2 \in \N}}
\newcommand{\sseq}[3]{\{#1_{#2_{#3}}\}_{#3 \in \N}}
\newcommand{\chep}{Let $\epsilon > 0$}

% Category Theory
\newcommand{\catC}{\mathcal{C}}
\newcommand{\catD}{\mathcal{D}}
\newcommand{\functF}{\mathcal{F}}
\newcommand{\functG}{\mathcal{G}}
\newcommand{\functI}{\mathcal{I}}
\newcommand{\functU}{\mathcal{U}}

\newcommand{\op}[1]{#1^{\mathrm{op}}}
\newcommand{\Obj}{\mathrm{Obj}}

\newcommand{\Set}{\mathbf{Set}}
\newcommand{\Grp}{\mathbf{Grp}}
\newcommand{\Top}{\mathbf{Top}}
\newcommand{\cRing}{\mathbf{cRing}}
\newcommand{\BanAnaMan}{\mathbf{BanAnaMan}}
\newcommand{\FinSet}{\mathbf{FinSet}}
\newcommand{\Vect}{\mathbf{Vect}}
\newcommand{\Two}{\mathbf{2}}


% ================= %
% Headers & Footers
% ================= %
\pagestyle{fancy}
\fancyhf{}
\newcommand{\intros}[3]{
  \lhead{\textbf{#1} {#2}}
  \rhead{#3}}
\rfoot{\thepage}
\renewcommand{\headrulewidth}{0pt}



% ================= %
%       Utils
% ================= %
\newcommand{\induction}[3]{
  \textbf{Base Case}. #1 \\
  \textbf{Inductive Hypothesis}. \\ #2 \\
  \textbf{Inductive Step}. \\ #3
}



% Used to list all problems on homework
\newcommand{\problems}[1]{
  \medskip \noin
  {\bf Problems}

  #1

  \medskip{}
}


% When prof does Question/Answer styling
\newcommand{\qna}[2]{
  {\bf Question}: #1

  {\bf Answer}: #2
}


% augmented matrices
\makeatletter
\renewcommand*\env@matrix[1][*\c@MaxMatrixCols c]{%
  \hskip -\arraycolsep
  \let\@ifnextchar\new@ifnextchar
  \array{#1}}
\makeatother


\newcommand\restr[2]{{% we make the whole thing an ordinary symbol
  \left.\kern-\nulldelimiterspace % automatically resize the bar with \right
  #1 % the function
  \littletaller % pretend it's a little taller at normal size
  \right|_{#2} % this is the delimiter
  }}

\newcommand{\littletaller}{\mathchoice{\vphantom{\big|}}{}{}{}}

% ================= %
%      Box Meta
% ================= %

% #2 - FG Color
% #3 - BG Color
\newenvironment{fancyleftbar}[3][\hsize]
{%
    \def\FrameCommand
    {%
        {\color{#2}\vrule width 3pt}%
        \hspace{0pt}%must no space.
        \fboxsep=\FrameSep\colorbox{#3}%
    }%
    \MakeFramed{\hsize#1\advance\hsize-\width\FrameRestore}%
}
{\endMakeFramed}

\newenvironment{simpleleftbar}[3][\hsize]
{%
    \def\FrameCommand
    {%
        {\vrule width 0.5pt}%
        \hspace{3pt}
        \fboxsep=\FrameSep%
    }%
    \MakeFramed{\hsize#1\advance\hsize-\width\FrameRestore}%
}
{\endMakeFramed}

% Used to allow the color argument to pass through the environment%
\newsavebox{\boxqed} 

% #1 - Header
% #2 - FG Color
% #3 - BG Color
\newenvironment{fancybox}[3]{
  \sbox\boxqed{\textcolor{#2}{$\blacksquare$}}
  \begin{fancyleftbar}{#2}{#3}

  \noin
  #1
  % {\large \bf \underline{#1}}
  \smallskip\noin \\
}
{

  \medskip
  \noin
  \usebox\boxqed

  \end{fancyleftbar}
}

% #1 - Text header
% #2 - Outer Text
% #3 - Inner Text
% #4 - Inner Header
% #5 - FG Color
% #6 - Background Color
\newcommand{\boxmeta}[6]{
  #1
  % {\small\sc\uppercase{#1}}

  #2

  \begin{fancybox}{#4}{#5}{#6}
    \noin
    #3
  \end{fancybox}
}

% #1 - Title
% #2 - FG Color
% #3 - BG Color
% #4 - Inner Text
\newcommand{\baronly}[4]{
  \begin{simpleleftbar}{#2}{#3}
    {\bf #1}.

    #4
  \end{simpleleftbar}
}

% ================= %
%     Box Colors
% ================= %

\definecolor{theorem_fg}{HTML}{EABAC3}
\definecolor{theorem_bg}{HTML}{F9EEF0}

\definecolor{problem_fg}{HTML}{ABABAB}
\definecolor{problem_bg}{HTML}{EDEDED}

\definecolor{lemma_fg}{HTML}{D0C97D}
\definecolor{lemma_bg}{HTML}{FCF9DB}

\definecolor{prop_fg}{HTML}{7DDB89}
\definecolor{prop_bg}{HTML}{D7FADB}

\definecolor{defn_fg}{HTML}{83D4CF}
\definecolor{defn_bg}{HTML}{E7FCFB}

\definecolor{lst_bg}{HTML}{EFF6F8}
\definecolor{lst_fg}{HTML}{475857}

\definecolor{btw_fg}{HTML}{5A5A5A}

% ================= %
%     Box Envs
% ================= %

\newcommand{\Definition}[2]{
                              \boxmeta{}{}{#2}{{\it Definition}. {\bf\underline{#1}}}{defn_fg}{defn_bg}
                            }

\newcommand{\Theorem}[2]{
  \boxmeta{{\bf Theorem.}}{#1}{#2}{{\bf Proof.}}{theorem_fg}{theorem_bg}
}

\newcommand{\NamedTheorem}[3]{
  \boxmeta{#1}{#2}{#3}{Proof}{theorem_fg}{theorem_bg}
}

\newcommand{\Problem}[3]{
  \boxmeta{Problem #1}{#2}{#3}{Solution}{problem_fg}{problem_bg}
}

\newcommand{\Example}[2]{
  \boxmeta{Example}{#1}{#2}{}{problem_fg}{problem_bg}
}

\newcommand{\Lemma}[2]{
  \boxmeta{{\bf Lemma.}}{#1}{#2}{{\bf Proof.}}{lemma_fg}{lemma_bg}
}

\newcommand{\NamedLemma}[3]{
  \boxmeta{#1}{#2}{#3}{Proof}{lemma_fg}{lemma_bg}
}

\newcommand{\Corollary}[2]{
  \boxmeta{Corollary}{#1}{#2}{Proof}{lemma_fg}{lemma_bg}
}

\newcommand{\Proposition}[2]{
  \boxmeta{Proposition}{#1}{#2}{Proof}{prop_fg}{prop_bg}
}

% ================== %
%      Bar Only
% ================== %

\newcommand{\definition}[1]{
  \baronly{Definition}{defn_fg}{defn_bg}{#1}
}

\newcommand{\theorem}[1]{
  \baronly{Theorem}{theorem_fg}{theorem_bg}{#1}
}

\newcommand{\example}[1]{
  \baronly{Example}{problem_fg}{problem_bg}{#1}
}

\newcommand{\remark}[1]{
  \baronly{Remark}{problem_fg}{problem_bg}{#1}
}

\newcommand{\note}[1]{
  \medskip
  \baronly{Note}{problem_fg}{problem_bg}{#1}
}

\newcommand{\btw}[1]{
  {\bf Author Note}.

  \begin{color}{btw_fg}
    #1
  \end{color}
}

\newcommand{\sidenote}[1]{
  {\bf Side Note}.

  \boxmeta{}{}{#1}{}{problem_fg}{problem_bg}
}



% graphs
\def\deg{\text{deg}}
\def\indeg{\text{indeg}}
\def\outdeg{\text{outdeg}}

% big O notation
\def\O{\mathcal O}

% define parent
\def\pr{\text{pr}}

% define incomplete commands
\def\TODO{\color{red}\textbf{TODO}\color{black}\,}
\def\QUESTION{\color{red}\textbf{QUESTION}\color{black}\,}



% listings settings
\lstset{
  % general styles
  backgroundcolor=\color{lst_bg},
  numbers=left,
  numberstyle=\color{lst_fg}\ttfamily\textbf,
  numbersep=3mm,
  frame=l,
  framesep=7mm,
  framexleftmargin=1.5mm,
  fillcolor=\color{lst_bg},
  rulecolor=\color{lst_bg},
  xleftmargin=9mm,
  % keyword styles
  keywordstyle=[1]\textbf,
  keywordstyle=[2]\textit,
  keywordstyle=[3]\textbf\textit,
  keywords=[1]{let, for, while, not, if, else, then, do, end, return},
  keywords=[2]{if, condition},
  keywords=[3]{do},
  mathescape=true, % enable math mode in listings
  columns=fullflexible,
  basicstyle=\ttfamily % this fond looks a bit better than the default
}





% colors used in proof boxes
\definecolor{proof_fg}{HTML}{ABABAB}
\definecolor{proof_bg}{HTML}{EDEDED}



% box solutions
\newenvironment{solution}{
  \begin{leftbar}

  \noin
  {\large \sc solution} \\
}
{

  \medskip
  \noin
  \textcolor{proof_fg}{$\blacksquare$}

  \end{leftbar}
}



% style a problem and solution
\newcommand{\prob}[3]{
  \bigskip \bigskip \noin
  {\Large\sc Problem #1}

  \medskip\noin
  #2

  \begin{solution}
    \noin
    #3
  \end{solution}
}

\renewcommand{\mod}[1]{
  \text{ (mod $#1$) }
}

\def\mat{\text{Mat}}
\def\gl{\text{GL}}
\def\sl{\text{SL}}



\begin{document}

\tableofcontents

\date{Wed. 24 Jan 2024}

\note {
All info for the class is available on the canvas page.
Notes from the prof are written on the iPad, and PDFs will be provided after
each class. Despite this taking notes is helpful.

Offices are hours are Tuesdays in person, and Thursdays on Zoom. Hours may
vary.

A text for this class is not {\it required}. Technically, we are using
Fraleigh's {\it A first course in Abstract Algebra}

No quizzes in this class, weekly Homeworks except on Exam weeks, two
midterms, and one final.

Readings on the class schedule are not additional, it's for people that need
extra material, or people that missed that day.
}

\section{Introduction - Sets and Relations}

\subsection{Sets}

\definition{
	A {\bf Set}. is a well-defined collection of objects called {\it elements}.

	$a \in A$ means ``$A$ is a set, $a$ is an element of a set, and $a$ is in
	$A$.
}

Examples of sets are

\begin{itemize}
	\item $\Z$ - The set of all integers, positive, negative, and zero
	\item $\N$ - The set of natural numbers, $0, 1, 2, \dots$. {\bf In this
			      class, $\N$ starts with $1$.}
	\item $\Q$ - The set of rational numbers.
	\item $\R$ - The set of real numbers.
	\item $\C$ - The set of complex numbers.
	\item $\{1, 2, 3, 4\}$
	\item $\{a \in \Z \mid a > 2 \}$. This is a set of integers {\it such that} $a
		      > 2$.
	\item $\varnothing$ - The empty set.
	\item $\text{GLn}(\R)$ - The set of $n \times n$ invertible matrices with real
	      entries. (GL stands for ``General Linear".)
	\item $C(\R)$ - The set of continuous functions $f: \R \to \R$.
\end{itemize}

\definition{
	A set $A$ is a {\bf subset} of a set $B$ if

	\[
		\forall x \in A, x \in B
	\]

	In other words, everything in $A$ is also in $B$. As notation, we can say
	either $A \subseteq B$ or $A \subset B$.

	A {\bf proper} subset is $A \subset B$ but $A \ne B$. Just write $A
		\subsetneq B$.
}

For example, $\N \subseteq \Z \subseteq \Q \subseteq \R \subseteq \C$. And,
importantly, $\varnothing \subseteq A$ for all sets $A$. In other words, the
empty set is a subset of {\it all} sets.

Two sets are equal if $A = B$, or $A \subseteq B$ {\it and} $B \subseteq A$.
This is often how you prove set equality.

\subsubsection{Set Operations}

We have four main operations.

\begin{itemize}
	\item {\bf Union}: $A \cup B = \{x \mid x \in A \text{ or } x \in B \}$.
	\item {\bf Intersection}: $A \cap B = \{ x \mid x \in A \text{ and } x \in B \}$.
	\item {\bf Product}: $A \times B = \{(a, b) \mid a \in A, b \in B \}$.
	\item {\bf Difference}: $A \setminus B = \{x \mid x \in A \text{ and } x \not\in B\}$
\end{itemize}

Two sets are {\bf disjoint} if $A \cap B = \varnothing$.

If we're working in a particular {\it universe} $U$ (i.e. all sets are subsets
of the universal set $U$) then the {\it complement} of $A$ is $A^c = \{x \mid x
	\in U \text{ and } x \not\in A \}$.

\subsection{Relations}

\definition {
	A {\bf relation} between sets $A$ and $B$ is a subset $R \subseteq A \times
		B$.
}

If $(a, b) \in R$, then we say that ``$a$ is related to $b$", or we write
$aRb$, or $a \sim b$.

\example {
	$R \subseteq \{1, 2, 3\} \times \{2, 3, 4\}$. $R = \{(1, 3), (2, 2), (3, 4) \}$
}

\note {
	Relations might not be reflexive! If $(a, b) \in R$, that means $a$ is
	related to $b$, but it might not be the case that $(b, a) \in R$. In other
	words, the reverse may not be true!
}

Another example might be $R \subseteq \R \times \R$, with $R = \{(x, x^3) \mid x
	\in \R \}$. Oh look! We just rewrote $f(x) = x^3$, so functions are relations.

\definition {
	A {\bf partition} of a set $A$ is a collection of {\it disjoint} subsets
	whose union is $A$.

	Another way to think of this is that any element of $A$ is in one and only
	one of its partitions.
}

An example of this might be the partition

\[
	A = \Z = \{x \in \Z \mid x < 0 \} \cup \{ 0 \} \cup \{ x \in \Z \mid x > 0 \}
\]

is a partition of $\Z$ into 3 sets.

Another example might be $A = \R$, subsets are $\{ x \}$ for each $x \in \R$.

Another, maybe more interesting example might be the following.

\example {
Fix $n \in \N$, $n \ge 2$. Let

\begin{itemize}
	\item $\bar 0 = \{x \in \Z \mid x \text{ is divisible by } n \}$.
	\item $\bar 1 = \{x \in \Z \mid x - 1 \text{ is divisible by } n \}$.
	\item $\bar 2 = \{x \in \Z \mid x - 2 \text{ is divisible by } n \}$. On and
	      on until...
	\item $\overline{n - 1} = \{x \in \Z \mid x - (n - 1) \text{ is divisible by } n
		      \}$.
\end{itemize}

{\bf Claim}: This partitions $\Z$ into in $n$ subsets.

}

\date{Fri. 26 Jan 2024}

Let's go back to relations, which we put aside to talk about partitions.

\definition {
	A relation $A \subseteq A \times A$ is called an {\bf equivalence relation}
	if it satisfies 3 properties

	\begin{enumerate}
		\item {\bf Reflexivity}: $aRA$ for all $a \in A$.
		\item {\bf Symmetry}: $aRb$ if and only if $bRa$.
		\item {\bf Transitivity}: If $aRb$ and $bRc$, then $aRc$.
	\end{enumerate}
}

The key idea is that equivalence relations on $A$ are {\it the same} as
partitions of $A$. What's going on here?

From an equivalence relation: If $b$ is related to $b$, put them in the same
set. Because of symmetry of equivalence relations, order of elements in the
set doesn't matter.

Conversely, given a partition say $aRb \Leftrightarrow bRa$ are in the same
subset.

\note {
	We'll be talking a lot about partitions and equivalence relations in this
	class.
}

Now we move on to the next step in our intro: functions.

\subsection{Functions}

A function $f: A \to B$ is a relation $R_f \subseteq A \times B$ such that,
for all $a \in A$, there is a {\it unique} $b \in B$ such that $aRb$.
Effectively, this means that

\begin{enumerate}
	\item We pass the vertical line test.
	\item The function is defined over its entire domain.
\end{enumerate}

Which are the properties we expect of functions!

\example {
	Let $f: \R \to \R$, with $R_f = \{(x, x^3) \mid x \in \R \}$. We write $f(a)$
	for the value $b$ where $(a, b) \in R_f$.
}

Given a function $f: A \to B$. We say that

\begin{itemize}
	\item $A$ is the {\it domain}.
	\item $B$ is the {\it codomain}.
	\item The {\it range} is a {\it subset} of the codomain, only where $f$
	      outputs values.
\end{itemize}

The $+$ operation is a function $+: \R \to \R$, also written as $(a, b)
	\mapsto a + b$. The multiplication operation $\times: \R \times \R \to \R$,
also written as $(a, b) \to ab$. These are binary operations, very useful in
Group Theory.

\definition {
	A {\bf binary operation} on a set $A$ is a function $f: A \times A \to A$.
	Its an operation on two inputs that outputs one thing.
}

\note {
	A dot product does not count here! Because the output of the dot
	product does not come from the same set as the input.
}

To do more complicated things in real life (such as $a + b + c$), we must
parenthesize.

\[
	f(a, f(b, c)) \text{ or } f(f(a, b), c)
\]

Of course this doesn't matter for addition in particular, but it might for
other binary operators!

\example {
	Fix $n \ge 2$ and consider $\Z_n = \{\bar 0, \bar 1, \dots, \overline{n - 1}
		\}$ (Note that this is a set of sets!)

	We want to come up with binary operations on this set. We have

	\begin{enumerate}
		\item Addition: $+: \Z_n \times \Z_n \to \Z_n$, defined as $(\bar a, \bar
			      b) \mapsto \overline{a + b}$.

		      But this isn't well-defined! For instance, what happens if $a + b$
		      exceeds $n$? To fix this, let's add the following condition:

		      Let $\bar x = \bar y$ if $x, y$ are in the same subset of partitions.
		      (i.e. They have the same remainder mod $n$.)

		      \qna {
			      Is this a well-defined binary operations?
		      }
		      {
			      Yes! But we must check that it doesn't matter how we define our
			      inputs.
		      }

		\item Multiplication: $\times: \Z_n \times \Z_n \to \Z_n$, defined as
		      $(\bar a, \bar b) \mapsto \overline{ab}$
	\end{enumerate}
}

\note {
	We also write $x \equiv y \mod{n} $ if $\bar x = \bar y$.
}

Now we're ready to jump in.

\section{Properties of Operations on R}

Let's look at the properties of $(\R, +)$ and $(\R \setminus \{0\}, \times)$.

\begin{multicols}{2}
	$(\R, +)$

	\begin{enumerate}
		\item {\bf Associativity}: $a + (b + c) = (a + b) + c$

		\item {\bf Identity}: $a + 0 = a$

		\item {\bf Inverses}: $a + (-a) = (-a) + a = 0$

		\item {\bf Commutativity}: $a + b = b + a$
	\end{enumerate}

	$(\R \setminus \{0\}, \times)$

	\begin{enumerate}
		\item {\bf Associativity}: $a \times (b \times c) = (a \times b) \times c$

		\item {\bf Identity}: $a \times 1 = a$.

		\item {\bf Inverses}: $a \times (1/a) = (1/a) \times a = 1$

		\item {\bf Commutativity}: $a \times b = b \times a$
	\end{enumerate}
\end{multicols}

\definition {
	We say that a binary operation $p: A \times A \to A$ is {\bf associative} if

	\[
		p(a, p(b, c)) = p(p(a, b), c)
	\]

	for any $a, b, c \in A$. In other words, how we parenthesize doesn't matter.
}

\definition {
	We say that a binary operation $p: A \times A \to A$ has an {\bf identity} if

	\[
		p(a, e) = p(e, a) = a
	\]

	for any $a \in A$.
}

\definition {
	We say that a binary operation $p: A \times A \to A$ has a {\bf inverses} if

	\begin{enumerate}
		\item It has an identity element $e$ (otherwise identity is meaningless!)

		\item
		      \[
			      p(a, b) = p(b, a) = e
		      \]

		      for any $a \in A$ and some $b \in A$.
	\end{enumerate}

	We usually write $b$ as $a^{-1}$.
}

\definition {
	We say that a binary operation $p: A \times A \to A$ is {\bf commutative} if

	\[
		p(a, b) = p(b, a)
	\]

	for any $a, b \in A$.
}

Let's look at properties of $(\Z_n, +)$ and $(\Z_n, \times)$.

$(\Z_n, +)$

\begin{enumerate}
	\item Is Associative.
	\item Has an identity: $0$.
	\item Has an inverse: $\overline{(-a)}$ for any $\bar a$.
	\item Is Commutative: We can move elements around.
\end{enumerate}

$(\Z_n, \times)$

\begin{enumerate}
	\item Is Associative
	\item Has an identity: $1$.
	\item {\bf Does not} have an inverse! Because $\bar 0$ is still there, we
	      have no inverse.
	\item Is Commutative: We can move elements around. In this case, $\bar a
		      \bar b = \overline{ab} = \overline{ba} = \bar b \bar a$.
\end{enumerate}

\qna {
	If we instead looked at $(\Z_n \setminus \{ 0 \}, \times)$, would there be
	inverses?
}
{
	We've messed the whole thing up! This is not even an binary operation
	anymore. Since we don't have $\bar 0$, what does $\bar 2 + \bar 2$ even mean
	now, if $n = 4$?

	We'll study this more in about a week.
}

Let's look at matrices. $A = \mat_n(\R)$ be the set of $n \times n$ matrices
with real elements, with the binary operation being matrix multiplication.
Let's look at its properties.

\begin{enumerate}
	\item Its associative.
	\item It has an identity.
	\item It {\bf does not} have an inverse.
	\item It {\bf is not} commutative.
\end{enumerate}

Now looking at $A = \gl_n(\R)$ be the set of $n \times n$ invertible matrices
with real entries.

\begin{enumerate}
	\item Its associative.
	\item It has an identity.
	\item It {\bf does} have an inverses.
	\item It {\bf is not} commutative.
\end{enumerate}

{\bf Proposition}

If $p: A \times A \to A$ is a binary operation with two identities $e, f$,
then $e = f$.

	{\bf Proof}

$e = p(e, f) = f$, so $e = f$.


	{\bf Proposition}

If we have two inverses, then they are the same. More formally: if $p(a, b) =
	p(b, a)= e$, and $p(a, c) = p(c, a) = e$, then $b = c$

{\bf Proof}

$p(c, p(a, b)) = p(c, e) = c$, but we could have also done $p(p(c, a), b) =
	p(e, b) = b$, so $b = c$.

\date{Mon. 29 Jan 2024}

\note {
	Homework 1 is due this Thursday at 11:59PM, on Gradescope. Because this is
	the first homework, Gradescope will allow late submissions but just submit
	it on time.
}

Last time, we talked about binary operations and their properties. Now, we are
going to put everything together and talk about Groups!

\section{Groups}

\definition {
	A {\bf Group} is a set $G$ with a binary operation $p: G \times G \to G$
	that

	\begin{enumerate}
		\item Is {\it Associative}.
		\item Has an {\it Identity}.
		\item Has {\it Inverses}.
	\end{enumerate}
}

Note that it does {\bf not} have commutatitivity. We'll talk about that
later.

Notation-wise, we write $(G, p)$, or just $G$ if the binary operations is
understood. Additionally, we often write the operation as $a \cdot b$, $a + b$,
or $ab$ instead of $p(a, b)$.

\definition {
	A Group is {\bf Abelian} if the operation is also {\it commutative}.
}

\note {
	Sometimes, we say that a group is {\it closed} under its operation. However
	we don't need this because a binary operation, by definition, is necessarily
	closed.
}

Let's look at some examples.

\example {
	These groups are {\bf Abelian}:
	\begin{enumerate}
		\item $(\R, +)$
		\item $(\Z, +)$
		\item $(\C, +)$
		\item $(\R \setminus \{0\}, \times)$
	\end{enumerate}

	These groups are {\bf Non-Abelian}:
	\begin{enumerate}
		\item $(\gl_n(\R), \times)$. Recall that this is the set of {\it
				      non-invertible} $n \times n$ matrices with real entries.
		\item $(\Z_n, +)$. Recall that this was the set of classes of partitions
		      modulo $n$.
	\end{enumerate}

	These are {\bf not Groups}:
	\begin{enumerate}
		\item $(\N \cup \{0\}, +)$, has no inverses.
		\item $\mat_n(\R), \times)$, has no inverses.
	\end{enumerate}
}

\definition {
	The {\bf Order} of a group, is the {\it cardinality} of the set $G$, denoted
	$|G|$.
}

\subsection{Cancellation Law}

In a group $G$, if $ab = ac$, then $b = c$.

	{\bf Proof}

Since $G$ has inverses, there is an element $a^{-1} \in G$ such that $a a^{-1} =
	e$. So,

\begin{align*}
	ab          & = ac                                    \\
	a^{-1} (ab) & = a^{-1} (ac)                           \\
	(a^{-1}a) b & = (a^{-1}a) c                           \\
	b           & = c           & \text{Defn of Identity} \\
\end{align*}

Let's look at some more examples.

\subsection{Groups of Matrices}

\example {
	From the groups of matrices, we can also talk about

	\begin{itemize}
		\item $\gl_n(\R)$
		\item $\gl_n(\C)$
		\item $\gl_n(\Q)$
	\end{itemize}

	Which are all groups.

		{\bf Question}: Is $\gl_n(\N)$ a group? What about $\gl_n(\Z)$?
}

Recall that $\gl$ stands for {\it general linear}. There is also the {\it
		special linear} group $\sl$. This is the set of general linear matrices with
determinant $1$. Let's look at some examples

\example {
	\begin{enumerate}
		\item $\sl = \{ A \in \gl_n(\R) \mid \det(A) = 1\}$

		      Recall that $\det(AB) = \det(A) \det(B)$, so this is closed.
	\end{enumerate}
}

\subsection{Symmetric Groups}

\definition {
	Given the set $\{1, 2, \dots, n\}$, the group of {\it permutations} of this
	set is the {\bf symmetric group} $S_n$, where the binary operation is {\it
			function composition}.

	A {\bf permutation} is a bijection $\{1, 2, \dots, n\} \to \{1, 2, \dots, n\}$.
	A permutation can be described by a list.
}

\example {
	If $n = 3$, we have the permutations

	\[
		\{123, 213, 132, 321, 231, 312 \}
	\]

	We say that $\sigma: \{1, 2, \dots, n\} \to \{1, 2, \dots, n\}$ takes an input
	from the set and defines the shuffle.
}

We'll talk more about Symmetric groups later in the semester.

\note {
	\begin{itemize}
		\item The Symmetric group is {\bf Non-Abelian}.
		\item There are $n!$ permutations of $S_n$, so the order of $S_n$ is $|S_n|
			      = n!$
		\item Why the Symmetric Group is named as it is is a question for another
		      day.
	\end{itemize}
}

\subsection{Subgroups}

\definition {
	A {\bf subgroup} is a subset $H$ of a group $(G, p)$ such that:

	\begin{enumerate}
		\item $H$ is closed under $p$.

		      If $a, b \in H$, then $p(a, b) \in H$.

		      Note that we {\it need} to explicitly state that a subgroup is closed,
		      because $p$ is {\bf not} closed in $H$, but it {\it is} by virtue of the
		      values in $H$. However this does not come for free from $p$, unlike with
		      $G$ like before.

		\item $H$ has inverses.

		      If $a \in H$, then $a^{-1} \in H$.
	\end{enumerate}

	Note that we also have

	\begin{enumerate}
		\item {\bf The Identity}, by virtue of $H$ being closed and containing
		      inverses.

		      \[
			      a a^{-1} = e
		      \]

		\item {\bf Associativity}, because $(G, p)$ is associative. This property is
		      just inherited from $G$.
	\end{enumerate}

	So $(H, p)$ is a group!
}

As notation, we say that $H \le G$ if $H$ is a subgroup of $G$.

\example {
	\[
		\sl_n(\R) \le \gl_n(\R) \le \gl_n(\C)
	\]

	Notice the direction of ``subset-ness"!
}

\example {
	\[
		\{\bar 0, \bar 2\} \le (\Z_4, +)
	\]

	Let's check this one.

	\begin{enumerate}
		\item {\bf Closure}:

		      This is small enough that we can check them all.

		      \begin{itemize}
			      \item $\bar 0 + \bar 0 = \bar 0$
			      \item $\bar 0 + \bar 2 = \bar 2$
			      \item $\bar 2 + \bar 0 = \bar 2$
			      \item $\bar 2 + \bar 2 = \bar 4 = \bar 0$
		      \end{itemize}

		      Note that we didn't really need to check the middle two, since the group
		      is Abelian, and that property is inherited.

		\item {\bf Inverses}

		      \begin{itemize}
			      \item $\bar 0 + \bar 0 = \bar 0$
			      \item $\bar 2 + \bar 2 = \bar 0$
		      \end{itemize}
	\end{enumerate}

	So we have a subgroup!
}

\example {
	If $G$ is a group and $a \in G$ is an element, then

	\[
		H = \{ \dots, a^{-3}, a^{-2}, a^{-1}, e, a^1, a^2, a^3, \dots \}
	\]

	is a subgroup.

	As a note

	\begin{itemize}
		\item $a^{-3} = a^{-1}a^{-1}a^{-1}$
		\item $a^2 = aa$
	\end{itemize}
}

Furthermore, sometimes, $a^n = e$ for some finite $n$. The smallest such $n$ is
called the {\bf order} of $a$.

\example {
	The order of $\bar 2 \in \Z_4$ is $2$.
}

\definition {
	We say that a subgroup $H \le G$ is called {\bf trivial} if $|H| = 1$. Or,

	\[
		H = \{e\}
	\]

	This is a subgroup of {\it every group}.
}

\note {
	$G \le G$ for all groups $G$. In other words, a group is always a subgroup of
	itself.

	We can say that $H < G$ is a {\bf proper} subgroup if $H \le G$ but $H \ne G$
	and, additionally for this class, $H$ is non-trivial.
}

\date{Wed. 31 Jan 2024}

Today we are going to talk about subgroups of $\Z$ under addition. We want to
understand {\it all} those subgroups. Both the techniques and the results will
be useful beyond just this set of groups.

Let $a \in \Z$, and let $a \Z = \{ ax \mid x \in \Z \}$ be all the multiples of
$a$ (with $0 \Z = \{ 0 \}$.)

{\bf Claim.}

$a \Z$ is a subgroup of $\Z$.

	{\bf Proof.}

We need check two properties.

\begin{enumerate}
	\item {\bf Closure}: Given $ax, ay \in a \Z$, $ax + ay = a(x + y) \in a\Z$.
	\item {\bf Inverses}: Given $ax \in a \Z$, $a(-x) \in a\Z$, and $ax + a(-x) =
		      ax - ax = 0 \in a\Z$.
\end{enumerate}

\note {
	This is how you should prove your questions relating to subgroups on the
	homework.
}

{\bf Claim}.

If $H \le \Z$ is a subgroup, then $H = a\Z$ for some $a \in \Z$. In other words,
this is it! This is {\it all} the subgroups.

	{\bf Proof}.

If $H \le \Z$, then $0 \in H$. If $H = \{ 0 \}$< then $H = 0\Z$ is the trivial
subgroup. Otherwise, $H$ contains non-zero integers. Since $H$ contains
inverses, it contains positive integers. Let $a$ be the smallest positive
integer in $H$. We want to show that $H = a\Z$.

Given $ax \in a\Z$, we can express $ax$ as follows

\[
	ax = \begin{cases}
		a + \cdots + a       & x > 0 \\
		0                    & x = 0 \\
		(-a) + \cdots + (-a) & x < 0 \\
	\end{cases}
\]

In all such cases, $H$ is closed and has inverses/identity, so $ax \in H$ and
thus $a\Z \subseteq H$.

The harder way is going backwards.

Given $h \in H$, and assume $|h| > a$ (We can do this because $a$ is the
smallest positive integer in $H$.) Write

\[
	h = ax + r
\]

Where $0 \le r < a$. We know that $h \in H$, and $ax \in H$, so

\[
	r = h - ax \in H
\]

Because $r$ is a combination of two elements in the subgroup! But recall that
$r$ is between $0$ and $a$. But we said before that $a$ is the smallest positive
integer in $H$, so $r$ {\it must} be zero! In other words $h = ax$ and $h \in
	a\Z$. Which proves that $H \subseteq a\Z$.

So $H = a\Z$.

\note {
	This proof is very important and the techniques in it come back! Be sure you
	understand what's going on.
}

This is great! We've now categorized every subgroup of the Integers under
addition!

Now, given $a\Z$, $b\Z$, form

\[
	a\Z + b\Z = \{ax + by \mid x, y \in \Z \}
\]

This is a subgroup of $\Z$. In fact,

{\bf Theorem}

% $a\Z + \b\Z = d\Z$ for some $d \in \Z$, $d \ge 0$. To prove this, we must first
% introduce some definitions.

\definition {
	If $a, b \ne 0$, then $d$ is the {\bf greatest common divisor} (gcd), of $a$
	and $b$,

	\[
		d = \gcd(a, b)
	\]
}

If $a, b = 0$, $d = \gcd(a, b)$, then

\begin{enumerate}
	\item $d$ divides $a$ and $b$, notated as $d \mid a$ and $d \mid b$.

		      {\bf Proof}.

	      $a \cdot 1 + b \cdot 0 = a \in d\Z$, so $d \mid a$. Similarly for $b$, $a \cdot
		      0 + b \cdot 1 = b \in d\Z$ so $d \mid b$.

	\item if $e \mid a$ and $e \mid b$, then $e \mid d$

	      {\bf Proof}

	      If $e \mid a$ $e \mid b$, then $e \mid (ax + by) = d$.

	\item $\exists x, y \in \Z$ such that $d = ax + by$

	      {\bf Proof}.

	      $d \in a\Z + b\Z$, so $d = ax + by$, for some $x, y \in \Z$.
\end{enumerate}

{\bf Fact}.

$d$ is the smallest positive value of $|ax + by|$.

This is useful, because if $ax + by = 1$ for some $x, y$, then $\gcd(a, b) = 1$.

\definition {
	$a, b \in \Z$ are {\bf relatively prime} if $\gcd(a, b) = 1$ and

	\[
		\gcd(a, b) = 1 \Leftrightarrow ax + by = 1
	\]

	for some $x, y \in \Z$.
}

{\bf Proposition}.

Let $p$ be a prime. If $p \mid ab$, then $p \mid a$ or $p \mid b$.

	{\bf Proof}.

Assume that $p$ is a prime, and $p \mid ab$, but $p \nmid a$.

We will show that $p \mid b$.

The factors of $p$ are $1$ and $p$, and $p \nmid a$, so $\gcd(p, a) = 1$ (since
the $\gcd$ is either $1$ or $p$, but if it was $p$, then $p$ would divide $a$.)

So there must exist $x, y \in \Z$ with $px + ay = 1$. Multiplying by $b$, we
have $pxb + aby = b$. Now of course, $p \mid pxb$ and, more importantly, $p \mid aby$
since $a$ is a multiple of $p$ so

\[
	p \mid (pxb + aby) = b
\]

So $p \mid b$.

	{\bf Similarly}.

$a\Z \cap b\Z$ is also a subgroup of $\Z$, say $m\Z$ and $m = \lcm(a, b)$, the
least common multiple of $a$ and $b$: The smallest number which is both a
multiple of $a$ and $b$.

	{\bf The Euclidean Algorithm}. {\it To find the $\gcd$}

To understand this, let's look at an

\example {
	Suppose we want to find the $\gcd(210, 45)$. Write $210 = 45 \cdot 4 + 30$. If
	$x \mid 210$ and $x \mid 45$, then $x \mid 30$. Now $x \mid 30$ and $x \mid 45$ implies that
	$x \mid 210$.

	Hence $\gcd(210, 45) = \gcd(45, 30)$.

	We can do this trick again!

	$45 = 30 \cdot 1 + 15$, so $\gcd(45, 30) = \gcd(30, 15) = 15$. So $\gcd(210,
		45) = 15$.
}

{\bf Cyclic Subgroups}.

$G$ is a group, and $a \in G$. The set

\[
	\langle a \rangle = \{\dots, a^{-3}, a^{-2}, a^{-1}, e, a, a^2, a^3, \dots\}
\]

is called the {\bf cyclic subgroup generated by $a$}.

\note {
	$\langle a \rangle$ is the smallest subgroup of $G$ that contains all these
	powers of $a$.

	$|\langle a \rangle| = |a|$, the smallest positive $n$ such that $a^n = e$, or
	$\infty$.
}

\date{Fri. 2 Feb 2024}

\section{Cyclic Groups and Subgroups}

We're going to repeat a little bit from last class, just to make sure we're on
the same page.

\definition {
	If $G$ is a group and $a \in G$, the set

	\[
		\lr {a} = \{\dots, a^{-3}, a^{-2}, a^{-1}, e, a, a^2 , a^3, \dots\}
	\]

	Is the cyclic subgroup generated by $a$.

	If $G = \lr{a}$ for some $a \in G$, we say $G$ is a {\bf cyclic group}.
}

\example {
	$G = \Z$, $a = 2$, We have that

	\[
		\lr {2} = \{\dots, -4, -2, 0, 2, 4, \dots\}
	\]

	In general, $\lr{a} = a\Z$, and $\lr {1} = \Z$, so $\Z$ is a cyclic group.
}

\note {
	$\lr {a}$ is the smallest subgroup of $G$ containing $a$.
}

Recall: Let $n \in \N$ be the smallest number such that $a^n = e$ (or $\infty$
if $a^n \ne e$ for all $n$) we say that $n$ is the order of $a$, or $|a| = n$.

	{\bf Proposition}

Let $|a| = n < \infty$. Then

\begin{enumerate}
	\item $a^l = a^m$ if and only if $l - m \equiv 0 \mod n$.
	\item $\lr {a} = \{e, a, a^2, \dots, a^{n - 1}\}$, and $|\lr{a}| = n$.
\end{enumerate}

{\bf Proof}

\begin{enumerate}
	\item If $a^l = a^m$, then $a^la^{-m} = e$ and so $a^{l - m} = e$. Assume that
	      $l - m \le 0$.

	      If $l - m \not\equiv 0 \mod n$ then $l - m > n$, because $n$ is minimal.
	      Then $l - m = nk + r$, and $r \in \{0, 1, \dots, n - 1\}$.

	      So

	      \[
		      a^r = a^{(l - m) - nk} = \underbrace{a^{l - m}}_{e} \underbrace{(a^n)^{-k}}_{e}
	      \]

	      But $r < n$ so in fact $r = 0$. This contradicts $l - m \not\equiv 0 \mod
		      n$, hence $l - m \equiv 0 \mod n$.

	\item If $l \in \Z$, write $l = nk + r$, $r \in \{0, 1, \dots, n - 1\}$, then
	      $a^l = a^{nk + r} = (a^n)^ka^r = e^ka^r = a^r$. So

	      \[
		      \lr{a} = \{e, a, \dots, a^{n - 1}\}
	      \]

	      If $a^l = a^m$ for $l, m \in \{0, 1, \dots, n - 1\}$, then $l - m \equiv 0
		      \mod n$. This only happens for $l = m$, so $|\lr{a}| = n$.
\end{enumerate}

This answers the question to the overloading of the word ``order" from before.
The {\it order} of an element is in fact the order of the cyclic subgroup that
it generates!

\example {
	If $|a| = n$, then

	\[
		|a^l| = \frac{n}{\gcd(n, l)}
	\]

	This is a good exercise for understanding subgroups. If you understand why
	it's true, you're in good shape.
}

\definition {
	An {\bf infinite cyclic group} is a cyclic group $\lr{a}$ where $|a| =
		\infty$.

	For example, $\Z$.

		{\bf Finite cyclic groups}, for example $\Z_n = \lr{ \bar 1 }$
}

\example {
	If $G = \gl_2(\R)$, then

	\[
		A = \begin{bmatrix}
			2 & 0 \\
			0 & 2 \\
		\end{bmatrix}
	\]

	Has infinite order. Raising it to a power keeps generating larger and larger
	matrices.

	However other matrices have finite order. For example, rotation matrices! In
	fact, it's possible to generate a rotation matrix of any order!

	\[
		B_n = \begin{bmatrix}
			\cos(2 \pi / n) & -\sin(2 \pi / n) \\
			\sin(2 \pi / n) & \cos(2 \pi / n)
		\end{bmatrix}
	\]

	Has order $n$. It's a rotation matrix!
}

\subsection{Homomorphisms}

So far we've been studying groups in isolation, but we may want to make general
statements about the relation between different groups.

We want function between groups that ``respect" the group operation.

\definition {
	Given groups $(G, p)$ and $(G', p')$. A {\bf Homomorphism} $\varphi: G \to G'$
	is a function such that

	\[
		\varphi(p(a, b)) = p'(\varphi(a), \varphi(b))
	\]

	It doesn't matter if we combine elements before or after the binary
	operations.

	Suppressing $p$ and $p'$, we can write $\varphi(ab) = \varphi(a) \varphi(b)$.
}

Alternatively, we have the following diagram

% https://q.uiver.app/#q=WzAsNCxbMCwwLCJHIFxcdGltZXMgRyJdLFsyLDAsIkciXSxbMCwyLCJHJyBcXHRpbWVzIEcnIl0sWzIsMiwiRyciXSxbMCwyLCJcXHZhcnBoaSBcXHRpbWVzIFxcdmFycGhpIiwxXSxbMCwxLCJwIiwxXSxbMiwzLCJwJyIsMV0sWzEsMywiXFx2YXJwaGkiLDFdXQ==
\[
	\begin{tikzcd}
		{G \times G} && G \\
		\\
		{G' \times G'} && {G'}
		\arrow["{\varphi \times \varphi}"{description}, from=1-1, to=3-1]
		\arrow["p"{description}, from=1-1, to=1-3]
		\arrow["{p'}"{description}, from=3-1, to=3-3]
		\arrow["\varphi"{description}, from=1-3, to=3-3]
	\end{tikzcd}
\]

\example {
	We can express the determinant function as

	\[
		\det : \gl_n(\R) \to (\R \setminus \{0\}, \times)
	\]

	Or

	\[
		A \to \det(A)
	\]

	And we can check that $\det(AB) = \det(A)\det(B)$
}

Similarly

\example {
	Consider the function $\exp : (\R, +) \to (\R \setminus \{0\}, \times)$

	Or

	\[
		x \to e^x
	\]

	And we can check that $\exp(x + y) = e^x e^y = \exp(x) \exp(y)$
}

A more general

\example {
	Given a group $G$, $a \in G$, we have

	\[
		\varphi: (\Z, +) \to G
	\]

	or

	\[
		n \to a^n
	\]

	If $|a| = n$, then $\varphi(\Z_n, +) \to G$, or $\bar i \to a^i$.
}

\example {
	The {\bf trivial homomorphism} $\phi: G \to G'$ can be defined as $a \to e$
	for all $a \in G$.
}

\note {
	The difference between an {\it Isomorphism} and a {\it Homomorphism} is {\bf
			not} necessarily a bijection.
}

{\bf Proposition}

If $\varphi: G \to G'$ is a homomorphism, then

\begin{enumerate}
	\item $\varphi(a_1, \dots, a_n) = \varphi(\a_1) \cdots \varphi(a_n)$
	\item $\varphi(e_G) = e_{G'}$
	\item $\varphi(a^{-1}) = \varphi(a)^{-1}$
\end{enumerate}

It's important to note that these aren't by definition, but {\it derived} from
the definition.

	{\bf Proof}.

\begin{enumerate}
	\item This one is by induction, we won't prove it.
	\item $\varphi(e_{G} e_{G}) = \varphi(e_G) \varphi(e_G)$

	      We can can ``cancel" $\varphi(e_G)$ from both sides and get

	      \[
		      e_{G'} = \varphi(e_G)
	      \]

	\item $e_{G'} = \varphi(e_G) = \varphi(a a^{-1}) = \varphi(a) \varphi(a^{-1})$
	      so $\varphi(a)\varphi(a^{-1}) = e_{G'}$ and so $\varphi(a^{-1}) =
		      \varphi(a)^{-1}$
\end{enumerate}

\date{Mon. 5 Feb 2024}

Recall: A {\bf Homomorphism} is a function $\phi: G \to G'$ satisfying $\phi(ab)
	= \phi(a)\phi(b)$.

When we talk about functions, we like to talk about the {\it image} of that
function.

\definition {
	The {\bf image} (or {\bf range}) of $\phi$ is $\phi(G) = \{\phi(a) \mid a \in
		G \}$ is the set of all outputs of $\phi$ on its domain.
}

\definition {
	The {\bf Kernel} of $\phi$ is $\ker(\phi) = \{a \in G \mid \phi(a) = e_{G'} \}$.
}

A vector space under addition is actually a group!

{\bf Proposition}

\begin{enumerate}
	\item $\phi(G)$ is a subgroup of $G'$. The image is a subgroup of the
	      codomain.

		      {\bf Closure}

	      If $\phi(a), \phi(b) \in \phi(G)$, then $\phi(a) \phi(b) = \phi(ab)$ since
	      $\phi$ is a Homomorphism. But notice that $\phi(ab) \in \phi(G)$, so we have
	      closure.

		      {\bf Inverses}

	      If $\phi(a) \in \phi(G)$, then $\phi(a)^{-1} = \phi(a^{-1}) \in \phi(G)$, so
	      it's a subgroup.

	\item $\ker(G)$ is a subgroup of $G$. The kernel is a subgroup of the domain.

		      {\bf Closure}

	      If $a, b \in \ker(\phi)$, $\phi(ab) = \phi(a) \phi(b) = e \cdot e = e$, so
	      $\phi(ab) \in \ker(\phi)$.

		      {\bf Inverses}

	      If $a \in \ker(\phi)$, then $\phi(a^{-1}) = \phi(a)^{-1} = e^{-1} = e$, so
	      $a^{-1} \in \ker(\phi)$.
\end{enumerate}

Let's look at an

\example {
	Consider the determinant function $\det: \gl_n(\R) \to (\R \setminus \{0\},
		\times)$, $A \to \det(A)$

	Here, the determinant is onto, so $\det(\gl_n(\R)) = \R \setminus \{0\}$.
	Additionally,

	\[
		\ker(\det) = \{A \in \gl_n(\R) \min \det(A) = 1 \} = \sl_n(\R)
	\]
}

To prove that something is a subgroup, it's often useful to find a homomorphism
whose kernel (or image) is a subgroup.

Consider the following

\example {
	$\exp(\R, +) \to (\R \setminus \{0\}, \times)$ is a homomorphism, $x \to e^x$.

	$\exp(\R) = \{x \in \R \mid x > 0\}$, and $\ker(\exp) = \{x \in \R \mid
		\exp(x) = 1 \} = \{0\}$.
}

{\bf Proposition}:

$\phi: G \to G'$ is one to one if and only if $\ker(\phi) = \{e\}$.

	{\bf Proof}:

($\Rightarrow$): $\phi(e) = e$, so $e \in \ker(\phi)$.

If $a \ne e$ and $a \in \ker(\phi)$, then $\phi(a) = \phi(e) = e$ so $\phi$ is
not injective, and we have a contradiction. Thus $\ker(\phi) = \{e\}$.

($\Leftarrow$): If $\phi(a) = \phi(b)$, then $\phi(a) \phi(b)^{-1} = e$, but
then $\phi(a) \phi(b^{-1}) = \phi(a b^{-1}) = e$ but since only $\phi(e) = e$,
this means $a b^{-1} = e$ and so $a = b$ and so $\phi$ is injective.

\subsection{Isomorphisms}

\definition {
	An {\bf Isomorphism} $\phi: G \to G'$ is a bijective homomorphism.
}

\note {
	To check that something is an Isomorphism, you need to check two things:

	\begin{enumerate}
		\item It's a Homomorphism
		\item It's a Bijection.

		      Note that if the function sets are finite, you only need to prove either
		      one to one-ness or onto-ness and the other should follow. Think about
		      why!
	\end{enumerate}
}

\subsubsection{Examples}

Some examples include:

\begin{itemize}
	\item $\exp: (\R, \times) \to (\R_{> 0}, \times)$
	\item $\phi': (\Z, +) \to \lr {a} \le G$ is an isomorphism {\bf if and only
			      if} $|a| = \infty$.

	      This should make sense, as we never get the identity by a non-zero power of
	      $a$.

	\item Given any $A \in \gl_n(\R)$, the linear map $f_A: (\R^n, +) \to (\R^n,
		      +)$ that sends $\bar x \to A \bar x$ is an isomorphism.

	\item If $a \in G$, the map $\phi_a: G \to G$ that sends $b \to aba^{-1}$ is
	      an isomorphism, called {\bf conjugation by $a$}

		      {\bf Check}:

	      \begin{enumerate}
		      \item {\bf Homomorphism}

		            \begin{align*}
			            \phi(bc) & = a(bc)a^{-1} = abeca^{-1} \\
			                     & = (aba^{-1})(aca^{-1})     \\
			                     & = \phi_a(b)\phi_a(c)
		            \end{align*}

		      \item {\bf One to One}:

		            If $\phi_a(b) = e$, then $aba^{-1} = e$, then $a^{-1}aba^{-1}a =
			            a^{-1}a$, and so $b = e$, so $\phi_a$ is injective.

		      \item {\bf Onto}:
		            If $c \in G$, we want $b \in G$ such that $\phi_a(b) = c$, i.e.
		            $aba^{-1} = c$.

		            So choose $b = a^{-1}ca$, then $\phi_a(b) = aba^{-1} = a(a^{-1}ca)a^{-1}
			            = c$, so $\phi_a$ is surjective.
	      \end{enumerate}
\end{itemize}

{\bf Proposition}: If $\phi: G \to G'$ is an isomorphism, then $\phi^{-1}: G'
	\to G$ is also an isomorphism.

	{\bf Proof}:

Since isomorphisms are bijections, it suffices to show that $\phi^{-1}$ is a homomorphism.

If $x, y \in G'$, we want to show that $\phi^{-1}(xy) = \phi^{-1}(x)
	\phi^{-1}(y)$.

Say $\phi^{-1}(x) = a$, $\phi^{-1}(y) = b$, and $\phi^{-1}(ab) = c$. Then we
want to show that $ab  = c$.

Then $ab = c$ if and only if $\phi(ab) = \phi(c)$ (since $\phi$ is bijective,
in fact injectivity is sufficient for this.)

But then this is true if and only if $\phi(a)\phi(b) = \phi(c)$, if and only if
$\phi(\phi^{-1}(x))\phi(\phi^{-1}(y)) = \phi(\phi^{-1}(xy))$, if and only if $xy
	= xy$. Thus $ab = c$ and $\phi^{-1}$ is a homomorphism.

\begin{align*}
	ab = c & \Leftrightarrow \phi(ab) = \phi(c)                                         \\
	       & \Leftrightarrow \phi(a)\phi(b) = \phi(c)                                   \\
	       & \Leftrightarrow \phi(\phi^{-1}(x))\phi(\phi^{-1}(y)) = \phi(\phi^{-1}(xy)) \\
	       & \Leftrightarrow xy = xy
\end{align*}

\definition {
	$G$ and $G'$ are {\bf isomorphic} if there exists an isomorphism $\phi: G \to
		G'$.
}

Note again that since isomorphisms are bijective, this means that the
isomorphism goes both ways.

We write $G \cong G'$ or $G \backsimeq G'$, but mostly the former.

\note {
	The goal of this class is to give a good classification to a lot of groups.
	Note that it's not really possible to completely do this, even for groups of a
	given order, because we have infinitely many possible groups of order one, but
	that's okay because there's an isomorphism between.

	We say that we care about these groups {\it up to isomorphism}.

	What this means is that if we have two groups that are isomorphic, we're going
	to treat them as the same. When classifying or counting, we can count $G$ and
	$G'$ are one if $G \cong G'$.
}

{\bf Exercise}. There is only one group of order one up to isomorphism.

So if $|G| = |G^{-1}| = 1$, then $G \cong G'$.

Next week we'll talk about specific types of groups and take a break from the
theory.

\date{Wed 07 Feb 2024}

\section{Important Groups}

\subsection{Groups mod n}

\[
	\Z_n = \{\bar 0, \bar 1, \dots, \overline{n - 1}\}
\]

Under $+$.

These are all cyclic, and in fact $\Z_n = \lr {\bar 1}$.

	{\bf Theorem}:

Any cyclic (sub)group is isomorphic to $\Z$ or $\Z_n$, for $n \ge 2$.

	{\bf Proof Idea}:

Write $f: \Z$ or $\Z$ to $\lr {a}$ which takes $k$ or $\bar k$ to $a^k$.

Chose $n = |a|$ or $\Z$ if $|a| = \infty$.

We know that subgroups of cyclic groups are cyclic (from Homework 1.) So any
subgroup of $\Z_n$ is $H = \lr{\bar m}$ for some $\bar m$.

\example {
	If $f | n$, then there exists some subgroup $H \le \Z_n$ with $|H| = f$.
}

\subsection{Multiplicative Groups}

In what context can we define a group under
multiplication for subsets of the integer $\mod n$?

If $\bar a$ is multiplicatively invertible $\mod n$, then there exists a
$\bar n \in \Z_n$ with $\bar a \cdot \bar b = \bar 1$.

We know that $(\Z_4, \times)$, for example, is not a group. So when {\it is
		it} a group?

We know that

\begin{align*}
	\bar a \cdot \bar b = \bar 1 & \Leftrightarrow ab \equiv 1 \mod n           \\
	                             & \Leftrightarrow \exists k \in \Z \text{ such
	that } ab = 1 + nk                                                          \\
	                             & \Leftrightarrow \exists k \in \Z \text{
	with } ab + n(-k) = 1                                                       \\
	                             & \Leftrightarrow \gcd(a, n) = 1
\end{align*}

Let's define $\Z_n^\times = \{\bar a \in \Z_n \setminus \{\bar 0\} \mid \gcd(a,
	n) = 1\}$

{\bf Theorem}:

$(\Z_n^\times, \times)$ is a group.

	{\bf Proof}: Omitted.

\example {
	$\Z_4^\times = \{\bar 1, \bar 3\}$, and looking at the multiplication table

	\begin{itemize}
		\item $\bar 1 \cdot \bar 1 = \bar 1$
		\item $\bar 1 \cdot \bar 3 = \bar 3$
		\item $\bar 3 \cdot \bar 1 = \bar 3$
		\item $\bar 3 \cdot \bar 3 = \bar 1$
	\end{itemize}

	And in fact, we see that $\Z_4^\times \cong (\Z_2, +)$, Where $\phi$ takes
	$\bar 1$ to $\bar 0$, and $\bar 3$ to $\bar 1$.
}

\example {
	Let's look at $\Z_8^\times = \{\bar 1, \bar 3, \bar 5, \bar 7\}$.

	In $\Z_8^\times$, every element $\bar a$ satisfies $\bar a^2 = \bar 1$, so
	$\Z_8^\times \cong \Z_4$, because in $\Z_4$, $\bar 1 + \bar 1 \ne \bar 0$ and
	$\bar 3 + \bar 3 = \bar 0$.

	So they are not isomorphic. If there were, say $f$, it would be
	surjective, we could pick $\bar a \in \Z_8^\times$ with $f(\bar a) = \bar 1$,
	then $f(\bar a \cdot \bar a) = f(\bar a) + f(\bar a)$ but this can't be
	the case.
}

{\bf Corollary}: $\Z_n^\times = \Z_n \setminus \{0\}$ if and only if $n$ is prime.

So $\Z_p^\times = \{\bar 1, \dots, \overline{p - 1} \}$ is a group under $\times$
when $p$ is prime.

\note {
  These are {\bf not} subgroups of $Z_n$, they have a {\it different} binary
  operation! These groups are under multiplication while $\Z_n$ is under
  addition.
}

\subsection{Symmetric Groups}

Recall the definition of $S_n$

\begin{align*}
  S_n &= \{\text{ all permutations of } \{1, \dots, n\}\} \\
      &= \{\text{ all bijections } \{1, \dots, n\} \to \{1, \dots, n\} \}
\end{align*}

Furthermore, recall that $S_n = n!$.

\example {
  Consider $S_4$, we say that $\sigma \in S_4$ is a function $\sigma: \{1, 2, 3,
  4\} \to \{1, 2, 3, 4\}$ with

  \begin{itemize}
    \item $1 \mapsto 2$
    \item $2 \mapsto 4$
    \item $3 \mapsto 3$
    \item $4 \mapsto 1$
  \end{itemize}
}

This is really cumbersome to write so instead, we write

\TODO{} Oh my god I can't tex this. See picture

Or,

\[
  \sigma = (124)(3) = (241)(3) = (3)(412)
\]

Because $1$, $2$, $4$ forms a cycle, and $3$ is a self loop. And this uniquely
describes the picture! Although this notation is compact, it's not unique.
This is known as {\it cycle notation}.

If the group $S_4$ is understood, we can write $(124)(3)$ as $124$. We have a
tendency to just leave off the 1-cycles.

\example {
  Let's do some multiplication.

  $\sigma^2 = (124)(124)$. To do this, we first draw the picture.

  \TODO{}

  And to do $\sigma^2$, we follow the arrows twice (we do $\sigma$ twice!)

  So $\sigma^2 = (142)$.
}

\example {
  Let $\sigma = (124)$, and $\tau = (12)(34)$, what is $\tau \sigma$ ?

  Again, we can draw the graphs.

  \TODO{}

  Note that order matters here! It may not always be the case that $\sigma \tau$
  equals $\tau \sigma$.

  So $\tau \sigma = (12)(34)(124) = (234)$.
}

\note {
  It should make sense that $(123) = (12)(23)$, and more generally,

  \[
    (a_1 a_2 \cdots a_k) = (a_1 a_2) (a_2 a_3) \cdots (a_{k - 1} a_k)
  \]
}

A 2-cycle $(a, b)$ is called a {\bf transposition}. Every $\sigma \in S_n$ can
be written as a product of transpositions.

\definition {
  $\sigma \in S$ is {\bf even} if it's the product of an even number of transpositions.

  $\sigma \in S_n$ is {\bf odd} if it's the product of an odd number of transpositions.
}

{\bf Proposition}:

No $\sigma \in S_n$ is both even and odd.

{\bf Proof}

(My idea here is that you can probably represent the permutation as a graph and
color the vertices and make some statement about where it takes you or something
like that.)

Next time!

What we get out of this is a map $\sgn: S_n \to (\{\pm 1, \times\})$ with

\[
  \sigma \to \begin{cases}
    1 & \sigma \text{ even} \\
    -1 & \sigma \text{ odd} \\
  \end{cases}
\]

{\bf Claim}: $\sgn$ is a homomorphism called the {\it signature} homomorphism.

\end{document}
